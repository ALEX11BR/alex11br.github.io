%	The MIT License (MIT)
%
%	Copyright (c) 2021 Jitin Nair
%
%	Permission is hereby granted, free of charge, to any person obtaining a copy
%	of this software and associated documentation files (the "Software"), to deal
%	in the Software without restriction, including without limitation the rights
%	to use, copy, modify, merge, publish, distribute, sublicense, and/or sell
%	copies of the Software, and to permit persons to whom the Software is
%	furnished to do so, subject to the following conditions:
%	
%	THE SOFTWARE IS PROVIDED "AS IS", WITHOUT WARRANTY OF ANY KIND, EXPRESS OR
%	IMPLIED, INCLUDING BUT NOT LIMITED TO THE WARRANTIES OF MERCHANTABILITY,
%	FITNESS FOR A PARTICULAR PURPOSE AND NONINFRINGEMENT. IN NO EVENT SHALL THE
%	AUTHORS OR COPYRIGHT HOLDERS BE LIABLE FOR ANY CLAIM, DAMAGES OR OTHER
%	LIABILITY, WHETHER IN AN ACTION OF CONTRACT, TORT OR OTHERWISE, ARISING FROM,
%	OUT OF OR IN CONNECTION WITH THE SOFTWARE OR THE USE OR OTHER DEALINGS IN
%	THE SOFTWARE.
%	
%
%-----------------------------------------------------------------------------------------------------------------------------------------------%

%----------------------------------------------------------------------------------------
%	DOCUMENT DEFINITION
%----------------------------------------------------------------------------------------

% article class because we want to fully customize the page and not use a cv template
\documentclass[a4paper,12pt]{article}

%----------------------------------------------------------------------------------------
%	FONT
%----------------------------------------------------------------------------------------

% % fontspec allows you to use TTF/OTF fonts directly
% \usepackage{fontspec}
% \defaultfontfeatures{Ligatures=TeX}

% % modified for ShareLaTeX use
% \setmainfont[
% SmallCapsFont = Fontin-SmallCaps.otf,
% BoldFont = Fontin-Bold.otf,
% ItalicFont = Fontin-Italic.otf
% ]
% {Fontin.otf}

%----------------------------------------------------------------------------------------
%	PACKAGES
%----------------------------------------------------------------------------------------
\usepackage{url}
\usepackage{parskip} 	

%other packages for formatting
\RequirePackage{color}
\RequirePackage{graphicx}
\usepackage[usenames,dvipsnames]{xcolor}
\usepackage[scale=0.9]{geometry}

%tabularx environment
\usepackage{tabularx}

%for lists within experience section
\usepackage{enumitem}

\newcolumntype{L}{>{\raggedright\arraybackslash}X} 
\newcolumntype{C}{>{\centering\arraybackslash}X} 
\newcolumntype{R}{>{\raggedleft\arraybackslash}X} 

%to prevent spillover of tabular into next pages
\usepackage{supertabular}
\usepackage{tabularx}
\newlength{\fullcollw}
\setlength{\fullcollw}{0.47\textwidth}

%custom \section
\usepackage{titlesec}				
\usepackage{multicol}
\usepackage{multirow}

%CV Sections inspired by: 
%http://stefano.italians.nl/archives/26
\titleformat{\section}{\Large\scshape\raggedright}{}{0em}{}[\titlerule]
\titlespacing{\section}{0pt}{10pt}{10pt}

%Setup hyperref package, and colours for links
\usepackage[unicode, draft=false]{hyperref}
\definecolor{linkcolour}{rgb}{0,0.2,0.6}
\hypersetup{colorlinks,breaklinks,urlcolor=linkcolour,linkcolor=linkcolour}

%for social icons
\usepackage{fontawesome5}

%debug page outer frames
%\usepackage{showframe}

%----------------------------------------------------------------------------------------
%	BEGIN DOCUMENT
%----------------------------------------------------------------------------------------
\begin{document}

% non-numbered pages
\pagestyle{empty} 

%----------------------------------------------------------------------------------------
%	TITLE
%----------------------------------------------------------------------------------------

% \begin{tabularx}{\linewidth}{ @{}X X@{} }
% \huge{Your Name}\vspace{2pt} & \hfill \emoji{incoming-envelope} email@email.com \\
% \raisebox{-0.05\height}\faGithub\ username \ | \
% \raisebox{-0.00\height}\faLinkedin\ username \ | \ \raisebox{-0.05\height}\faGlobe \ mysite.com  & \hfill \emoji{calling} number
% \end{tabularx}

\begin{tabularx}{\linewidth}[]{@{} X c R @{}}
& \Huge{Ioan Alexandru Popa} &
%\begin{tabular}[c]{c}
%    \small
%    Availability for the internship: \\
%    \small
%    24th Jun - 22nd Sep 2024 \\
%\end{tabular}
\\[7.5pt]
\multicolumn{3}{@{}c@{}}{
    \small
    \href{https://github.com/ALEX11BR}{\raisebox{-0.05\height}\faGithub\ ALEX11BR} \ $|$ \ 
    \href{https://alex11br.github.io}{\raisebox{-0.05\height}\faGlobe \ alex11br.github.io} \ $|$ \ 
    \href{mailto:alexioanpopa11@gmail.com}{\raisebox{-0.05\height}\faEnvelope \ alexioanpopa11@gmail.com} \ $|$ \ 
    \href{https://www.linkedin.com/in/ioan-alexandru-popa-3a5597257/}{\raisebox{-0.05\height}{\faLinkedin}\ LinkedIn} \ $|$ \ 
    \href{tel:+40765081127}{\raisebox{-0.05\height}{\faPhone*}\ +40 765 081 127}
} \\
\end{tabularx}

%----------------------------------------------------------------------------------------
%	EDUCATION
%----------------------------------------------------------------------------------------
\section{Education}
\begin{tabularx}{\linewidth}{ @{}l r@{} }
Politehnica University of Bucharest, \textbf{Bachelor in Computer Science} & \hfill \faCalendar* Oct 2022 - Jul 2026 \\[3.75pt]
\multicolumn{2}{@{}X@{}}{
    \begin{minipage}[t]{\linewidth}
        \begin{itemize}[nosep,after=\strut, leftmargin=1em, itemsep=3pt]
            \item Relevant Coursework: Computer programming (\textbf{C} \& \textbf{Python}; \textbf{Assembly}; \textbf{Rust}), Operating systems usage, Data structures and algorithms, Numerical methods (\textbf{Octave}), Applied informatics 2 -- information processing (\textbf{Matlab}), Object-oriented programming (\textbf{Java}), Operating systems
            \item \textbf{Cumulative grades} (1st year): 9.18/10
        \end{itemize}
    \end{minipage}
}
\end{tabularx}

%\begin{tabularx}{\linewidth}{ @{}l X@{} }
%\textbf{``Nicolae Bălcescu'' National College}, Brăila & \hfill \faCalendar* 2018 - 2022 \\[3.75pt]
%\end{tabularx}

%----------------------------------------------------------------------------------------
%	PUBLICATIONS
%----------------------------------------------------------------------------------------
%Projects
\section{Projects}
\begin{tabularx}{\linewidth}{ @{}l r@{} }
\textbf{emscripten-functions} (\href{https://github.com/ALEX11BR/emscripten-functions}{GitHub link}) & \hfill \faCalendar* Aug 2023 - Oct 2023 \\[3.75pt]
\multicolumn{2}{@{}X@{}}{
    \begin{minipage}[t]{\linewidth}
        \begin{itemize}[nosep,after=\strut, leftmargin=1em, itemsep=3pt]
            \item Implemented raw \textbf{Rust} bindings for \textbf{emscripten} system functions. Emscripten is a compiler toolchain that allows C \& C++ code to be ran on web pages using \textbf{WASM}. Emscripten is the easiest way of building Rust code for web that depends on C or C++ libraries.
            \item Built Rust-friendly wrappers for \textbf{29} emscripten-specific functions.
            \item Over \textbf{175} downloads on \texttt{crates.io} for the \texttt{emscripten-functions} crate (the Rust-friendly wrappers), and over \textbf{215} for \texttt{emscripten-functions-sys} (the raw Rust bindings).
        \end{itemize}
    \end{minipage}
}
\end{tabularx}

\begin{tabularx}{\linewidth}{ @{}l r@{} }
\textbf{MyRustRoguelike} (\href{https://github.com/ALEX11BR/MyRustRoguelike}{GitHub link}) & \hfill \faCalendar* Jul 2022 - Nov 2022 \\[3.75pt]
\multicolumn{2}{@{}X@{}}{
    \begin{minipage}[t]{\linewidth}
        \begin{itemize}[nosep,after=\strut, leftmargin=1em, itemsep=3pt]
            \item Built a simple roguelike game in \textbf{Rust}, where the player has to descend through the \textbf{26} levels of a procedurally generated dungeon while dodging various enemies.
            \item Designed a common back-end, used by \textbf{3} different front-ends: ncurses, SDL, and web using \textbf{Yew}.
        \end{itemize}
    \end{minipage}
}
\end{tabularx}

\begin{tabularx}{\linewidth}{ @{}l r@{} }
\textbf{ThemeChanger} (\href{https://github.com/ALEX11BR/ThemeChanger}{GitHub link}) & \hfill \faCalendar* Aug 2021 - Jul 2022 \\[3.75pt]
\multicolumn{2}{@{}X@{}}{
    \begin{minipage}[t]{\linewidth}
        \begin{itemize}[nosep,after=\strut, leftmargin=1em, itemsep=3pt]
            \item Designed a Linux app in \textbf{Python} and \textbf{GTK3} that lets the user modify the mouse cursor, application icon, and widget themes and settings of \textbf{4} theme frameworks: GTK2, GTK3, GTK4, Kvantum.
            \item Implemented live theme reloading using \textbf{6} desktop environment-specific mechanisms.
            \item Built a mechanism of showing instantly GTK3 theme and CSS changes in the app.
        \end{itemize}
    \end{minipage}
}
\end{tabularx}

%\begin{tabularx}{\linewidth}{ @{}l r@{} }
%\textbf{fizoscomp} (\href{https://github.com/ALEX11BR/fizoscomp}{GitHub link}) & \hfill \faCalendar* Dec 2020 - Jan 2021 \\[3.75pt]
%\multicolumn{2}{@{}X@{}}{
%    \begin{minipage}[t]{\linewidth}
%        \begin{itemize}[nosep,after=\strut, leftmargin=1em, itemsep=3pt]
%            \item Designed a web app that displays oscillations with given amplitudes, phases and angular frequencies, and shows their sum, using \textbf{React} and \textbf{TypeScript}.
%            \item Implemented a responsive mobile-first design using \textbf{Bootstrap}.
%        \end{itemize}
%    \end{minipage}
%}
%\end{tabularx}

%----------------------------------------------------------------------------------------
%	SKILLS
%----------------------------------------------------------------------------------------
\section{Skills}
\begin{tabularx}{\linewidth}{@{}l X@{}}
\multicolumn{2}{@{}X@{}}{
    \begin{minipage}[t]{\linewidth}
        \begin{itemize}[nosep,after=\strut, leftmargin=1em, itemsep=3pt]
            \item \textbf{Intermediate}: C, C++, Linux \& shell scripting, Python, Rust, TypeScript
            \item \textbf{Basic}: CSS, C\#, Emscripten, F\#, Git, Go, Godot Engine, GTK3, HTML, Java, \LaTeX, Lua, Matlab/Octave, React, SQL, Svelte, x86 Assembly
        \end{itemize}
    \end{minipage}
} 
\end{tabularx}

\section{Extracurricular activities}
\begin{tabularx}{\linewidth}{@{}l r@{}}
\textbf{Security Summer School}, binary track & \hfill \faCalendar* Jun 2023 - Jul 2023 \\[3.75pt]
\multicolumn{2}{@{}X@{}}{
    \begin{minipage}[t]{\linewidth}
        \begin{itemize}[nosep,after=\strut, leftmargin=1em, itemsep=3pt]
            \item Attended lectures and solved exercises about finding and exploiting common executable vulnerabilities.
            \item Took part in 2 team-based Capture the Flag events where given executables had to be exploited to find a hidden ``flag''.
        \end{itemize}
    \end{minipage}
} 
\end{tabularx}

\section{Volunteer activities}

\begin{tabularx}{\linewidth}{@{}l r@{}}
\textbf{Rust workshop December 2023}, CLI applications track & \hfill \faCalendar* Dec 2023 \\[3.75pt]
\multicolumn{2}{@{}X@{}}{
    \begin{minipage}[t]{\linewidth}
        \begin{itemize}[nosep,after=\strut, leftmargin=1em, itemsep=3pt]
            \item Helped teach \textbf{29} students over a weekend the basics of \textbf{Rust} programming and CLI app development.
        \end{itemize}
    \end{minipage}
} 
\end{tabularx}

%\begin{tabularx}{\linewidth}{@{}l r@{}}
%\textbf{Rust workshop April 2023}, desktop applications track & \hfill \faCalendar* Apr 2023 \\[3.75pt]
%\multicolumn{2}{@{}X@{}}{
%    \begin{minipage}[t]{\linewidth}
%        \begin{itemize}[nosep,after=\strut, leftmargin=1em, itemsep=3pt]
%            \item Taught \textbf{2} students over a weekend the basics of desktop application development using \textbf{Tauri}.
%        \end{itemize}
%    \end{minipage}
%} 
%\end{tabularx}

%\vfill
%\center{\footnotesize Last updated: \today}

\end{document}
