%	The MIT License (MIT)
%
%	Copyright (c) 2021 Jitin Nair
%
%	Permission is hereby granted, free of charge, to any person obtaining a copy
%	of this software and associated documentation files (the "Software"), to deal
%	in the Software without restriction, including without limitation the rights
%	to use, copy, modify, merge, publish, distribute, sublicense, and/or sell
%	copies of the Software, and to permit persons to whom the Software is
%	furnished to do so, subject to the following conditions:
%
%	THE SOFTWARE IS PROVIDED "AS IS", WITHOUT WARRANTY OF ANY KIND, EXPRESS OR
%	IMPLIED, INCLUDING BUT NOT LIMITED TO THE WARRANTIES OF MERCHANTABILITY,
%	FITNESS FOR A PARTICULAR PURPOSE AND NONINFRINGEMENT. IN NO EVENT SHALL THE
%	AUTHORS OR COPYRIGHT HOLDERS BE LIABLE FOR ANY CLAIM, DAMAGES OR OTHER
%	LIABILITY, WHETHER IN AN ACTION OF CONTRACT, TORT OR OTHERWISE, ARISING FROM,
%	OUT OF OR IN CONNECTION WITH THE SOFTWARE OR THE USE OR OTHER DEALINGS IN
%	THE SOFTWARE.
%
%
%-----------------------------------------------------------------------------------------------------------------------------------------------%

%----------------------------------------------------------------------------------------
%	DOCUMENT DEFINITION
%----------------------------------------------------------------------------------------

% article class because we want to fully customize the page and not use a cv template
\documentclass[a4paper,12pt]{article}

%----------------------------------------------------------------------------------------
%	FONT
%----------------------------------------------------------------------------------------

% % fontspec allows you to use TTF/OTF fonts directly
% \usepackage{fontspec}
% \defaultfontfeatures{Ligatures=TeX}

% % modified for ShareLaTeX use
% \setmainfont[
% SmallCapsFont = Fontin-SmallCaps.otf,
% BoldFont = Fontin-Bold.otf,
% ItalicFont = Fontin-Italic.otf
% ]
% {Fontin.otf}

%----------------------------------------------------------------------------------------
%	PACKAGES
%----------------------------------------------------------------------------------------
\usepackage{url}
\usepackage{parskip}

%other packages for formatting
\RequirePackage{color}
\RequirePackage{graphicx}
\usepackage[usenames,dvipsnames]{xcolor}
\usepackage[scale=0.9]{geometry}

%tabularx environment
\usepackage{tabularx}

%for lists within experience section
\usepackage{enumitem}

\newcolumntype{L}{>{\raggedright\arraybackslash}X}
\newcolumntype{C}{>{\centering\arraybackslash}X}
\newcolumntype{R}{>{\raggedleft\arraybackslash}X}

%to prevent spillover of tabular into next pages
\usepackage{supertabular}
\usepackage{tabularx}
\newlength{\fullcollw}
\setlength{\fullcollw}{0.47\textwidth}

%custom \section
\usepackage{titlesec}
\usepackage{multicol}
\usepackage{multirow}

%CV Sections inspired by:
%http://stefano.italians.nl/archives/26
\titleformat{\section}{\Large\scshape\raggedright}{}{0em}{}[\titlerule]
\titlespacing{\section}{0pt}{10pt}{10pt}

%Setup hyperref package, and colours for links
\usepackage[unicode, draft=false]{hyperref}
\definecolor{linkcolour}{rgb}{0,0.2,0.6}
\hypersetup{colorlinks,breaklinks,urlcolor=linkcolour,linkcolor=linkcolour}

%for social icons
\usepackage{fontawesome5}

%debug page outer frames
%\usepackage{showframe}

%----------------------------------------------------------------------------------------
%	BEGIN DOCUMENT
%----------------------------------------------------------------------------------------
\begin{document}

% non-numbered pages
\pagestyle{empty}

%----------------------------------------------------------------------------------------
%	TITLE
%----------------------------------------------------------------------------------------

% \begin{tabularx}{\linewidth}{ @{}X X@{} }
% \huge{Your Name}\vspace{2pt} & \hfill \emoji{incoming-envelope} email@email.com \\
% \raisebox{-0.05\height}\faGithub\ username \ | \
% \raisebox{-0.00\height}\faLinkedin\ username \ | \ \raisebox{-0.05\height}\faGlobe \ mysite.com  & \hfill \emoji{calling} number
% \end{tabularx}

\begin{tabularx}{\linewidth}[]{@{} X c R @{}}
& \Huge{Ioan Alexandru Popa} &
%\begin{tabular}[c]{c}
%    \small
%    Availability for the internship: \\
%    \small
%    22nd Jun - 25th Sep 2025 \\
%\end{tabular}
\\[7.5pt]
\multicolumn{3}{@{}c@{}}{
    \small
    \href{https://github.com/ALEX11BR}{\raisebox{-0.05\height}\faGithub\ ALEX11BR} \ $|$ \
    \href{https://alex11br.github.io}{\raisebox{-0.05\height}\faGlobe \ alex11br.github.io} \ $|$ \
    \href{mailto:alexioanpopa11@gmail.com}{\raisebox{-0.05\height}\faEnvelope \ alexioanpopa11@gmail.com} \ $|$ \
    \href{https://www.linkedin.com/in/ioan-alexandru-popa/}{\raisebox{-0.05\height}{\faLinkedin}\ LinkedIn} \ $|$ \
    \href{tel:+40765081127}{\raisebox{-0.05\height}{\faPhone*}\ +40 765 081 127}
} \\
\end{tabularx}

%----------------------------------------------------------------------------------------
%	EDUCATION
%----------------------------------------------------------------------------------------
\section{Education}
\begin{tabularx}{\linewidth}{ @{}l r@{} }
Politehnica University of Bucharest, \textbf{Bachelor in Computer Science} & \hfill \faCalendar* Oct 2022 - Jul 2026 \\[3.75pt]
\multicolumn{2}{@{}X@{}}{
    \begin{minipage}[t]{\linewidth}
        \begin{itemize}[nosep,after=\strut, leftmargin=1em, itemsep=3pt]
            \item Relevant Coursework: Computer programming (\textbf{C} \& \textbf{Python}; \textbf{Assembly}; \textbf{Rust}), Data structures and algorithms, Numerical methods (\textbf{Octave}), Object-oriented programming (\textbf{Java}), Operating systems, Algorithm design, Communication protocols, Local computer networks, Introduction in cybersecurity
            \item \textbf{Cumulative grades}: 9.39/10
        \end{itemize}
    \end{minipage}
}
\end{tabularx}

%\begin{tabularx}{\linewidth}{ @{}l X@{} }
%\textbf{``Nicolae Bălcescu'' National College}, Brăila & \hfill \faCalendar* 2018 - 2022 \\[3.75pt]
%\end{tabularx}

%----------------------------------------------------------------------------------------
%	PUBLICATIONS
%----------------------------------------------------------------------------------------
%Projects
\section{Projects}

\begin{tabularx}{\linewidth}{ @{}l r@{} }
\textbf{emscripten-functions} (\href{https://github.com/ALEX11BR/emscripten-functions}{GitHub link}) & \hfill \faCalendar* Aug 2023 - Sep 2024 \\[3.75pt]
\multicolumn{2}{@{}X@{}}{
    \begin{minipage}[t]{\linewidth}
        \begin{itemize}[nosep,after=\strut, leftmargin=1em, itemsep=3pt]
            \item Implemented raw \textbf{Rust} bindings for \textbf{emscripten} system functions. Emscripten is a compiler toolchain that allows C \& C++ code to be ran on web pages using \textbf{WASM}. Emscripten is the easiest way of building Rust code for web that depends on C or C++ libraries.
            \item Built Rust-friendly wrappers for \textbf{29} emscripten-specific functions.
            \item Over \textbf{2600} downloads on \texttt{crates.io} for the \href{https://crates.io/crates/emscripten-functions}{\texttt{emscripten-functions}} crate (the Rust-friendly wrappers), and over \textbf{2300} for \href{https://crates.io/crates/emscripten-functions-sys}{\texttt{emscripten-functions-sys}} (the raw Rust bindings).
        \end{itemize}
    \end{minipage}
}
\end{tabularx}

\begin{tabularx}{\linewidth}{ @{}l r@{} }
\textbf{Personal dotfiles management system} (\href{https://github.com/ALEX11BR/dotfiles}{GitHub link}) & \hfill \faCalendar* Mar 2021 - Sep 2024 \\[3.75pt]
\multicolumn{2}{@{}X@{}}{
    \begin{minipage}[t]{\linewidth}
        \begin{itemize}[nosep,after=\strut, leftmargin=1em, itemsep=3pt]
            \item Created a GitHub repository with configuration files for select programs, like \texttt{vim}, \texttt{zsh}, VSCode.
            \item Implemented \textbf{shell scripts} for \textbf{5} distro families that install some apps and configure a freshly installed \textbf{Linux} system matching the personal preferences, and a \textbf{Windows} script that does this too.
        \end{itemize}
    \end{minipage}
}
\end{tabularx}

\begin{tabularx}{\linewidth}{ @{}l r@{} }
\textbf{ThemeChanger} (\href{https://github.com/ALEX11BR/ThemeChanger}{GitHub link}) & \hfill \faCalendar* Aug 2021 - Jul 2024 \\[3.75pt]
\multicolumn{2}{@{}X@{}}{
    \begin{minipage}[t]{\linewidth}
        \begin{itemize}[nosep,after=\strut, leftmargin=1em, itemsep=3pt]
            \item Designed a Linux app in \textbf{Python} and \textbf{GTK3} that lets the user modify the mouse cursor, application icon, and widget themes and settings of \textbf{4} theme frameworks, even for unthemable libadwaita apps.
            \item Implemented live theme reloading using \textbf{6} desktop environment-specific mechanisms.
            \item Built a mechanism of showing instantly GTK3 theme and CSS changes in the app.
        \end{itemize}
    \end{minipage}
}
\end{tabularx}

%\begin{tabularx}{\linewidth}{ @{}l r@{} }
%\textbf{Magnetic field mapping} & \hfill \faCalendar* Nov 2023 - Jan 2024 \\[3.75pt]
%\multicolumn{2}{@{}X@{}}{
%    \begin{minipage}[t]{\linewidth}
%        \begin{itemize}[nosep,after=\strut, leftmargin=1em, itemsep=3pt]
%            \item Designed a system of collecting magnetic field data from a ICM29048 connected to a Raspberry Pi Pico W using \textbf{I2C}.
%            \item Implemented a solution that sends the data to a computer that then processes the data, given user input about where the sensor currently is.
%        \end{itemize}
%    \end{minipage}
%}
%\end{tabularx}

%\begin{tabularx}{\linewidth}{ @{}l r@{} }
%\textbf{MyRustRoguelike} (\href{https://github.com/ALEX11BR/MyRustRoguelike}{GitHub link}) & \hfill \faCalendar* Jul 2022 - Nov 2022 \\[3.75pt]
%\multicolumn{2}{@{}X@{}}{
%    \begin{minipage}[t]{\linewidth}
%        \begin{itemize}[nosep,after=\strut, leftmargin=1em, itemsep=3pt]
%            \item Built a simple roguelike game in \textbf{Rust}, where the player has to descend through the \textbf{26} levels of a procedurally generated dungeon while dodging various enemies.
%            \item Designed a common back-end, used by \textbf{3} different front-ends: ncurses, SDL, and web using \textbf{Yew}.
%        \end{itemize}
%    \end{minipage}
%}
%\end{tabularx}

%\begin{tabularx}{\linewidth}{ @{}l r@{} }
%\textbf{MyFSharpSnake} (\href{https://github.com/ALEX11BR/MyFSharpSnake}{GitHub link}) & \hfill \faCalendar* Jan 2022 - Mar 2022 \\[3.75pt]
%\multicolumn{2}{@{}X@{}}{
%    \begin{minipage}[t]{\linewidth}
%        \begin{itemize}[nosep,after=\strut, leftmargin=1em, itemsep=3pt]
%            \item Built a Snake game clone in \textbf{F\#}.
%            \item Implemented a terminal version and a web one.
%        \end{itemize}
%    \end{minipage}
%}
%\end{tabularx}

%\begin{tabularx}{\linewidth}{ @{}l r@{} }
%\textbf{gxhk} (\href{https://github.com/ALEX11BR/gxhk}{GitHub link}) & \hfill \faCalendar* Dec 2021 - Jan 2022 \\[3.75pt]
%\multicolumn{2}{@{}X@{}}{
%    \begin{minipage}[t]{\linewidth}
%        \begin{itemize}[nosep,after=\strut, leftmargin=1em, itemsep=3pt]
%            \item Built using \textbf{Go} a hotkey daemon for \textbf{X11} configurable on-the-fly using \textbf{unix sockets}.
%        \end{itemize}
%    \end{minipage}
%}
%\end{tabularx}

\begin{tabularx}{\linewidth}{ @{}l r@{} }
\textbf{Proportional election methods simulation} (\href{https://github.com/ALEX11BR/proportional-election-simulation}{GitHub link}) & \hfill \faCalendar* Jan 2021 \\[3.75pt]
\multicolumn{2}{@{}X@{}}{
    \begin{minipage}[t]{\linewidth}
        \begin{itemize}[nosep,after=\strut, leftmargin=1em, itemsep=3pt]
            \item Implemented a \textbf{Svelte} web app that can simulate \textbf{4} proportional election apportionment methods.
        \end{itemize}
    \end{minipage}
}
\end{tabularx}

%\begin{tabularx}{\linewidth}{ @{}l r@{} }
%\textbf{fizoscomp} (\href{https://github.com/ALEX11BR/fizoscomp}{GitHub link}) & \hfill \faCalendar* Dec 2020 - Jan 2021 \\[3.75pt]
%\multicolumn{2}{@{}X@{}}{
%    \begin{minipage}[t]{\linewidth}
%        \begin{itemize}[nosep,after=\strut, leftmargin=1em, itemsep=3pt]
%            \item Designed a web app that displays oscillations with given amplitudes, phases and angular frequencies, and shows their sum, using \textbf{React} and \textbf{TypeScript}.
%            \item Implemented a responsive mobile-first design using \textbf{Bootstrap}.
%        \end{itemize}
%    \end{minipage}
%}
%\end{tabularx}

%\section{Extracurricular activities}

%\begin{tabularx}{\linewidth}{@{}l r@{}}
%\textbf{Security Summer School}, binary track & \hfill \faCalendar* Jun 2023 - Jul 2023 \\[3.75pt]
%\multicolumn{2}{@{}X@{}}{
%    \begin{minipage}[t]{\linewidth}
%        \begin{itemize}[nosep,after=\strut, leftmargin=1em, itemsep=3pt]
%            \item Attended lectures and solved exercises about finding and exploiting common executable vulnerabilities.
%            \item Took part in 2 team-based Capture the Flag events where given executables had to be exploited to find a hidden ``flag''.
%        \end{itemize}
%    \end{minipage}
%}
%\end{tabularx}

\section{Volunteer activities}

\begin{tabularx}{\linewidth}{@{}l r@{}}
\textbf{Undergraduate teaching assistant}, Assembly programming & \hfill \faCalendar* Feb 2024 - Present \\[3.75pt]
\multicolumn{2}{@{}X@{}}{
    \begin{minipage}[t]{\linewidth}
        \begin{itemize}[nosep,after=\strut, leftmargin=1em, itemsep=3pt]
            \item Held a weekly laboratory session for over \textbf{10} students that shows them the basics of \textbf{x86 Assembly}.
            \item Made a \textbf{Docker} image for lecture demos.
        \end{itemize}
    \end{minipage}
}
\end{tabularx}

\begin{tabularx}{\linewidth}{@{}l r@{}}
\textbf{Rust workshop April 2024}, desktop applications track & \hfill \faCalendar* Apr 2024 \\[3.75pt]
\multicolumn{2}{@{}X@{}}{
    \begin{minipage}[t]{\linewidth}
        \begin{itemize}[nosep,after=\strut, leftmargin=1em, itemsep=3pt]
            \item Taught \textbf{4} students over a Saturday day the basics of desktop application development using \textbf{Tauri}.
        \end{itemize}
    \end{minipage}
}
\end{tabularx}

%\begin{tabularx}{\linewidth}{@{}l r@{}}
%\textbf{Rust workshop December 2023}, CLI applications track & \hfill \faCalendar* Dec 2023 \\[3.75pt]
%\multicolumn{2}{@{}X@{}}{
%    \begin{minipage}[t]{\linewidth}
%        \begin{itemize}[nosep,after=\strut, leftmargin=1em, itemsep=3pt]
%            \item Helped teach \textbf{29} students over a weekend the basics of \textbf{Rust} programming and CLI app development.
%        \end{itemize}
%    \end{minipage}
%}
%\end{tabularx}

%\begin{tabularx}{\linewidth}{@{}l r@{}}
%\textbf{Rust workshop April 2023}, desktop applications track & \hfill \faCalendar* Apr 2023 \\[3.75pt]
%\multicolumn{2}{@{}X@{}}{
%    \begin{minipage}[t]{\linewidth}
%        \begin{itemize}[nosep,after=\strut, leftmargin=1em, itemsep=3pt]
%            \item Taught \textbf{2} students over a weekend the basics of desktop application development using \textbf{Tauri}.
%        \end{itemize}
%    \end{minipage}
%}
%\end{tabularx}

%\section{Open source contributions}
%\begin{itemize}[nosep,after=\strut, leftmargin=1em, itemsep=3pt]
%    \item Added my ThemeChanger application to nixpkgs. \hfill \faCalendar* Oct 2021 - Jul 2024
%    \item Enhanced the tests suite of Lemmy and \verb|rust-vmm/kvm-ioctls|. \hfill \faCalendar* Jan 2024
%    \item Implemented the option to hide the ``mentions'' tab in the Tiny IRC client. \hfill \faCalendar* Jan 2023
%    %\item Fixed typos in OpenTTD and Unikraft. \hfill \faCalendar* Mar 2022 - Oct 2023
%\end{itemize}

%----------------------------------------------------------------------------------------
%	SKILLS
%----------------------------------------------------------------------------------------
\section{Skills}
\begin{tabularx}{\linewidth}{@{}l X@{}}
\multicolumn{2}{@{}X@{}}{
    \begin{minipage}[t]{\linewidth}
        \begin{itemize}[nosep,after=\strut, leftmargin=1em, itemsep=3pt]
            \item \textbf{Intermediate}: C, C++, JavaScript, Linux \& shell scripting, Python, Rust, TypeScript
            \item \textbf{Basic}: CSS, C\#, Emscripten, F\#, Git, Go, Godot Engine, GTK3, HTML, Java, \LaTeX, Lua, Matlab/Octave, Racket/Scheme, React, SQL, Svelte, x86 Assembly
        \end{itemize}
    \end{minipage}
}
\end{tabularx}

%\vfill
%\center{\footnotesize Last updated: \today}

\end{document}
